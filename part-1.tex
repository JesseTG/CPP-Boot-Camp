%!TEX options = -shell-escape

\documentclass[glossy]{beamer}
\usepackage[utf8]{inputenc} % Unicode support (Umlauts etc.)
\usepackage{graphicx} % Add pictures to your document
\usepackage{hyperref} % Add a link to your document
\usepackage{minted}
\usepackage{tikz}
\usepackage{lmodern}
\usepackage[T1]{fontenc}

\useoutertheme{wuerzburg}
\useinnertheme[realshadow,corners=2pt,padding=2pt]{chamfered}
\usecolortheme{shark}

\setbeamertemplate{navigation symbols}{}

\usetikzlibrary{tikzmark, arrows, decorations, decorations.pathreplacing}
\newminted{cpp}{autogobble, fontsize=\tiny, escapeinside=@@}
\newmintinline{cpp}{}
\newmintinline{java}{}
\newmintinline{js}{}
\usemintedstyle{vs}

\newcommand<>{\hover}[1]{\uncover#2{%
 \begin{tikzpicture}[remember picture,overlay]%
 \draw[fill,opacity=0.4] (current page.south west)
 rectangle (current page.north east);
 \node at (current page.center) {#1};
 \end{tikzpicture}}
}

\tikzset{every picture/.style={font issue=\scriptsize},
         font issue/.style={execute at begin picture={#1\selectfont}}
        }


\title{C++ Boot Camp 1/2}
\author{Jesse Talavera-Greenberg}
\date{\today}

% Concepts to cover:
% Memory management
% Pointers and references
% Scope vs. lifetime
% const-correctness
% RAII
% OOP
% Rule of three
\begin{document}

\newcommand{\cppref}[2]{\href{http://en.cppreference.com/w/cpp/#1}{\underline{#2}}}

\begin{frame}
\maketitle
\end{frame}

\begin{frame}
\frametitle{This Week}
  \begin{itemize}
    \item No need to download anything this week
    \item Go to \href{www.cppreference.com}{www.cppreference.com} to follow code
    \item This is your C++ Bible this semester
    \item Online compiler
  \end{itemize}
\end{frame}

\begin{frame}
\frametitle{Disclaimer}
    \begin{itemize}
        \item I am not the grader for this course
        \begin{itemize}
            \item Undergrads can't grade assignments
        \end{itemize}
        \item The boot camp is always part of the course, but I volunteered to lead it this year.
        \item Any opinions expressed are my own and are not necessarily condoned by McKenna.
        \item I don't know very much about DirectX.
    \end{itemize}
\end{frame}

\begin{frame}
  \frametitle{Today}
  \begin{itemize}
    \item Core language features
    \item Memory management
    \item Gotchas
    \item How to do common things
  \end{itemize}
\end{frame}

\begin{frame}
  \frametitle{Distinct Points}
  \begin{itemize}
    \item Balance high-level language constructs and low-level resource management
    \item Huge-ass language
    \item Designed for C compatibility
    \item Accumulated lots of cruft over the years
  \end{itemize}
\end{frame}

\begin{frame}[fragile=singleslide]
  \frametitle{Basic Structure}
  \begin{cppcode}
#include <iostream>

using std::cout;
using std::endl;
// Now do this to use names (or prefix them with "std::")

// Main entry point of a C++ program
int |\cppref{language/main_function}{main}|(int argc, char** argv, char** env) {
  // All these args are optional, third one is not as common
  // argc == number of command-line arguments 
  // argv == the actual arguments as an array of strings
  // env == environment variables
  cout << "Hello!  My name is " << argv[0] << endl;

  return 0;
  // Return code of 0 tells the OS the program exited normally
  // Anything else == something went wrong (you decide what the numbers mean)
  // Leaving it off implies "return 0"
}
  \end{cppcode}
\end{frame}

\begin{frame}[fragile=singleslide]
  \frametitle{if, else, for, while}
  \begin{cppcode}
if (somethingIsTrue()) {
  for (int i = 0; i < 10; ++i) {
    break; // Exactly the same
  }
}
else if (somethingElseIsTrue()) {
  while (true) break; // You can omit braces for single statements like in Java
}
else {
  do {
    continue; // Exactly the same
  } while (false);
}
  \end{cppcode}
\end{frame}

\begin{frame}[fragile=singleslide]
  \frametitle{switch}
  \begin{cppcode}
switch (someInteger) { // All okay
case 0:
  break;
case SOME_CONSTANT:
  break;
default:
  // nop
}

switch (someString) { // ERROR
case "No way":
case "Too bad":
case "Don't bother":
default:
  // nop
}
  \end{cppcode}
\end{frame}

\begin{frame}[fragile=singleslide]
  \frametitle{Exceptions}
  \begin{cppcode}
using std::exception;

try {
  functionThatMightThrowException(); // No checked exceptions in C++
  throw exception("No new statement; we'll come back to this");
}
catch (exception e) {
  std::cout << e.what(); // Print the exception info
}

try {
  throw 5; // You can throw anything in C++
} catch (int i) {
  // Stick to exception objects and their subclasses, though
} catch (...) {
  // Catch anything, but you can't access it; can only be used last
}
  \end{cppcode}
\end{frame}

\begin{frame}[fragile=singleslide]
  \frametitle{Fundamental Types}
  \begin{itemize}
    \item Sizes are not well-defined like in Java
    \item No \verb|byte| (use \cppinline|uint8_t|)
    \item \verb|boolean| to \cppinline|bool|
    \item Use \cppinline|nullptr| instead of \verb|null| or \cppinline|NULL|
    \item Signed and unsigned types
    \item signed types are the same
    \item unsigned types double the max value, but can't represent negative numbers
    \item Need a specific number size?
    \item First \cppinline|#include <cstdint>|
    \item Then use \verb!std::u?int(8|16|32|64)_t!
  \end{itemize}
\end{frame}

\begin{frame}[fragile=singleslide]
  \frametitle{typedef and typeid}
  \begin{cppcode}
#include <cstdint>
#include <iostream>
#include <typeinfo>
// ^ Must include typeinfo to use the typeid operator

typedef std::uint8_t byte; // Declares another name for the same type
// Syntax:
//    typedef existing_type new_name;
using std::cout;
using std::endl; 

int main() { 
  std::uint8_t a = 5; 
  byte b = a; 
  // Totally okay; the distinction is compile-time only 

  cout << typeid(std::uint8_t).name() << endl; 
  cout << typeid(byte).name() << endl; 
  // Both printouts will be implementation-defined, but identical 

  /* Common standard typedefs: 
        size_t: Sizes of objects (usually an unsigned int)
        ptrdiff_t: Difference between pointers (usually an unsigned int or long) 
        std::string: Convenient form of std::basic_string<char> 
  */ 
}
  \end{cppcode}
\end{frame}

\begin{frame}[fragile=singleslide]
  \frametitle{Declaring Variables}
  \begin{cppcode}
#include <string>
#include <vector>

using std::string;
using std::vector;

int main() {
  string empty; // default constructor
  string empty2(); // default constructor with parens
  string value = "I'm really a char*";
  string value2("Parens are fine, too");
  string copy = value; // Copy assignment
  string& ref = value; // Reference assignment (points to value)
  const string& constref = value; // Constant (read-only) reference assignment
  // All of an object's members are initialized!

  string* ptr; // Unassigned pointer; not necessarily nullptr
  vector<int> ints = {1, 2, 3, 4, 5, 6}; // Initializer list
}
  \end{cppcode}
\end{frame}

\begin{frame}[fragile=singleslide]
  \frametitle{Undefined Behavior}
  \begin{itemize}
    \item Whenever something happens in Java, you can rely on something else to happen
    \item C++ lets the compiler do anything it wants in certain situations
    \item For performance reasons, mostly
    \item Many, many ways to do this
    \item Do not rely on undefined behavior
  \end{itemize}
\end{frame}

\begin{frame}[fragile=singleslide]
  \frametitle{Some examples}
  \begin{cppcode}
#include <string>
#include <cstdint> 
#include <limits> 

int32_t something(int32_t param) { 
  if (param > 0) return param; 
  // No return down here; now what? 
} 

std::string& returnsALocal() { 
  std::string reference = "Why are you doing this"; 
  return reference; // Uh-oh 
} 

int main() { 
  int32_t a = std::numeric_limits<int32_t>::max(); 
  a++; // Oops 

  int32_t* pointer; // What's inside?  Beats me.
  int32_t number = *pointer; // Crash (or worse; no crash)

  int32_t[10] array; 
  number = array[12]; // WTF 
}
  \end{cppcode}
\end{frame}

\begin{frame}[fragile=singleslide]
  \frametitle{Namespaces}
  \begin{cppcode}
#include <iostream>
#include <string>
#include <array>
#include <utility>

using namespace std;
// ^ Don't do this for the same reason you don't do java.util.*

namespace sbcs { // Nest them as much as you'd like
  namespace cse380 {
    struct Game {
      string name;
      string teamName;
      array<string, 3> members;
    };
  }

  namespace cse381 {
    typedef pair<string, string> Team;
  }
}

int main() {
  sbcs::cse380::Game game; // Fully-qualified type name
  using sbcs::cse381::Team; // Or use it in the current namespace (global in this case)

  Team team("Jesse", "Karl");
}
  \end{cppcode}
\end{frame}

\begin{frame}[fragile=singleslide]
  \frametitle{The Preprocessor}
  \begin{cppcode}
#include <iostream>
// Use <angle_brackets> for standard library/third-party code, "quotes" for your own 

#define DEBUG 
// You can also add #defines through your build system (e.g. the command line or 
// your Visual Studio project file) 

// #define WINDOWS 
// #define MAC 
// ^ Uncomment one of these 

using std::cout; 
using std::endl; 

int main() { 
  #ifdef DEBUG 
  cout << "Logging! Or sanity checking!" << endl;
  #endif 

  #ifdef WINDOWS
  cout << "Use DirectX!" << endl; 
  #else 
  cout << "Use OpenGL!" << endl; 
  #endif 

  #ifdef MAC 
    #error "Jesse doesn't like Apple"
    // ^ Intentionally causes a compiler error
  #endif
}
  \end{cppcode}
\end{frame}

\begin{frame}[fragile=singleslide]
  \frametitle{assert/static\_assert}
  \begin{cppcode}
#include <cassert>
#include <type_traits>

// #define NDEBUG
// ^ Uncomment this or define it in the build system to disable asserts

using std::is_base_of;
using std::is_same;

template<class T, class U>
void failIfUnrelated() {
  static_assert(
    is_base_of<T, U>::value
    || is_base_of<U, T>::value
    || is_same<T, U>::value,
    "Both types should be the same, or one should be a parent of the other"
  );
}

class Parent {};
class Child : public Parent {};

int main() {
  assert(1 + 1 == 2);
  assert(true && "Idiom for providing informative text in an assert()");

  failIfUnrelated<int, int>(); // Okay
  failIfUnrelated<int, Child>(); // Causes a compiler error, as desired
  failIfUnrelated<Parent, Child>(); // Okay
}
  \end{cppcode}
\end{frame}

\begin{frame}[fragile=singleslide]
  \frametitle{enums}
  \begin{cppcode}
#include <iostream> 

using std::cout; 
using std::endl; 

enum class Direction { 
  North, 
  South, 
  East,
  West, // Trailing comma is okay 
}; 

// NOT OBJECTS; just ints with some compile-time meaning 
enum class Speakers { 
  Mute = 0, // Assigning values is okay 
  Mono = 1, 
  Stereo = 2, 
  Surround = 4 
}; 

int main() { 
  Direction dir = Direction::North;
  Speakers audio = Speakers::Stereo;
  audio = Speakers(4); // Make sure int is a valid Speakers value!
  cout << int(audio) << endl;
  // int <-> enum class conversions must be explicit
}
  \end{cppcode}
\end{frame}

\begin{frame}[fragile=singleslide]
  \frametitle{Pointers}
  \begin{cppcode}
#include <string> 

using std::string; 

int main() { 
  string a = "I'm a little teapot"; 
  string* ptr = &a; // &some_object returns the address of some_object 
  string b = *ptr; // *some_pointer dereferences 

  int ints[] = {1, 2, 3, 4, 5, 6, 7, 8, 9, 10}; 
  // Raw arrays are pointers 

  int* int_ptr = ints; // Points to ints[0] 
  int* int_ptr2 = ints + 1; // Points to ints[1] 
                            // Increments the address sizeof(int) bytes 
  int** double_array = &int_ptr;
  // Pointers to pointers okay; never usually need more than two**

  void* pointer_to_anything = ints;
  // Pointer to arbitrary data; very error-prone and unsafe; avoid unless you know
  // exactly why you need one

  pointer_to_anything = nullptr;
  // nullptr == Java's null; assignable to any pointer type, but not to objects

  float* garbage;  // Uninitialized pointer; value is undefined
                   // Could be garbage, could be nullptr; initialize it ASAP!
}
  \end{cppcode}
\end{frame}

\begin{frame}[fragile=singleslide]
  \frametitle{References}
  \begin{cppcode}
#include <string> 
#include <iostream> 

using namespace std; // Only for slide space constraints 

class StringHolder { 
  public:
    StringHolder(const string& str) : _str(str) {}

    const string& getStringReference() const { return this->_str; }

    string getStringCopy() const { return this->_str; }

    void append(const string& str) { this->_str += str; }
    // ^ Passes an unmodifiable reference rather than a copy
  private:
    string _str;
};

int main() {
  StringHolder holder("France");
  const string& a = holder.getStringReference();
  string b = holder.getStringCopy();

  cout << (a == b) << endl; // True; comparing object values
  cout << (&a == &b) << endl; // False; different addresses, different objects

  holder.append(b);
  cout << holder.getStringReference() << endl; // FranceFrance
}
  \end{cppcode}
\end{frame}

\begin{frame}[fragile=singleslide]
  \frametitle{Pointers Vs. References (similarities)}
  \begin{itemize}
    \item Access large objects without copying them
    \item Left dangling (invalid) if the referenced object is deallocated
    \item No way to check for this
    \item Will likely cause a crash if you try to use it
    \item Both can point to an instance of the given type's superclass
    \item Both compiled to similar machine code
    \item Multiple pointers/references can refer to one object
  \end{itemize}
\end{frame}

\begin{frame}[fragile=singleslide]
  \frametitle{Pointers Vs. References (differences)}
  \begin{itemize}
    \item Can represent absence of data
    \item More prone to becoming invalid
    \item Syntax needed to convert between pointers and objects
    \item Can be assigned to different objects
    \item Always initialized with a valid object
    \item No syntax needed to use references aside from declarations or method signatures
    \item Always points to one location in memory
    \item Easier to deal with objects that shouldn't change
  \end{itemize}
\end{frame}

\begin{frame}[fragile=singleslide]
  \frametitle{new/delete}
  \begin{cppcode}
#include <string>

using std::string;

void wasteMemory() {
  string stack_allocated = "I disappear when I go out of scope";
  string* wasted = new string("I only disappear when explicitly deallocated");

  string* cleaned = new string("I'm about to get cleaned up properly");

  delete cleaned;
}

int main() {
  for (int i = 0; i < 10000; i++) {
    wasteMemory();
  }

  // stack_allocated disappeared when wasteMemory() returned
  // All instances of wasted are still in memory
  // cleaned disappeared when "delete cleaned;" was called
  // See: Scope vs. Lifetime
}
  \end{cppcode}
\end{frame}

\begin{frame}[fragile=singleslide]
  \frametitle{Exceptions}
  \begin{cppcode}
#include <iostream>
#include <string>
#include <exception>
#include <array>

using namespace std; // Again, slide space constraints

int main() {
  array<int, 5> ints = {1, 2, 3, 4, 5};

  try {
    cout << ints.at(10) << endl; // at() checks bounds, operator[] doesn't
  }
  catch (const out_of_range& e) {
    // You can (and should) catch by reference
    cout << "Out of range in an array" << endl;
  }
  catch (const runtime_error& e) {
    cout << "Didn't see this coming: " << e.what() << endl;
  }
  catch (const exception& e) {
    cout << "Unknown exception: " << e.what() << endl;
  }
  catch (...) {
    cout << "WTF?" << endl;
  }
  // There is no finally clause; use destructors to guarantee exception safety
}
  \end{cppcode}
\end{frame}

\begin{frame}[fragile=singleslide]
  \frametitle{Declaring Classes}
  \begin{cppcode}
// public/protected/private semantics same as in Java
// no specifier == private in classes, public in structs

class ItHoldsThings {
  public: 
    int cantBeOverridden(int a, int b) {
      // Technically you can, but things get weird
      return this->_integer + b;
    }

    virtual float canBeOverridden(const double d) {
      // virtual == overrideable
      return d * 2;
    }

    virtual void abstractMethod() = 0;

  protected:
    float _number;

  private:
    int _integer;
}; // Don't forget the semicolon
// When a class is constructed, ALL of its members are too
  \end{cppcode}
\end{frame}

\begin{frame}[fragile=singleslide]
  \frametitle{Constructors}
  \begin{cppcode}
#include <string>

using std::string; 

class StringWrapper { 
  public: 
    StringWrapper() { /* other code here */ } // _str == ""
    // Default constructor can be auto-generated with some caveats (see the wiki)

    StringWrapper(const string& value) : _str(value) {} // _str == value
    // Copy constructor can be auto-generated (but see wiki)
    // Base class constructors are called like this, too

  private:
    string _str;
    // All members are initialized on construction, whether you like it or not
};

int main() { 
  StringWrapper a;
  StringWrapper b("There will be a copy of me in c");
  StringWrapper c(b);
}
  \end{cppcode}
\end{frame}

\begin{frame}[fragile=singleslide]
  \frametitle{Destructors}
  \begin{cppcode}
// Using constructors to allocate resources and destructors to clean them up is 
// called Resource Acquisition Is Initialization (RAII). Only needed for low-level 
// resources; high-level classes like std::iostream do RAII themselves. 

#include <cstdlib> 

using std::malloc;
using std::free;

class LotsOfResources {
  public:
    LotsOfResources() {
      _floats = new float[64];
      _buffer = malloc(64);
    }

    ~LotsOfResources() {
      // Called when this object's lifetime ends delete[] _floats;
      // Remember the [] when deleting a raw array free(_buffer);
    }

  private:
    float* _floats;
    void* _buffer;
};

int main() {
  LotsOfResources* resources = new LotsOfResources;
  delete resources; // No memory leaks!
}
  \end{cppcode}
\end{frame}

\begin{frame}[fragile=singleslide]
  \frametitle{new/delete}
  \begin{cppcode}
  \end{cppcode}
\end{frame}

\begin{frame}[fragile=singleslide]
  \frametitle{Abstract classes and inheritance}
  \begin{cppcode}
#include <exception> 
#include <iostream> 

// There are no interfaces; use abstract classes without any concrete methods 
struct ProgramCrasher { 
  virtual void crash() = 0; // "= 0" means the method is abstract 
  // virtual == method can be overridden 
  // only one abstract method needed to define a class as abstract
}; 

struct IPastryChef { 
  virtual void bake() = 0; 
};

struct SegfaultCrasher : public ProgramCrasher {
  // Actually several types of inheritance; it's complicated, just use public

  void crash() override {
    int* a;
    int b = *a;
  }
};
// Multiple inheritance; don't do it if more than one parent class is not an interface

struct ExceptionCrasher : public ProgramCrasher, public IPastryChef {
  void crash() override {
    throw std::exception();
  }

  void bake() override {
    std::cout << "This is just my day job\n";
  }
};
  \end{cppcode}
\end{frame}

\begin{frame}[fragile=singleslide]
  \frametitle{Casting}
  \begin{cppcode}
#include <iostream>
#include <string>

using namespace std; // Slide space constraints 

struct Base { 
  virtual ~Base() {}
  string baseMethod() { return "base"; }
};

struct Derived : Base {
  virtual string derivedMethod() { return "derived"; }
};

int main() {
  Base* b1 = new Derived;
  Derived* d1_s = dynamic_cast<Derived*>(b1);
  cout << d1_s << ' ' << d1_s->baseMethod() << ' ' << d1_s->derivedMethod() << endl; 
  // Checking at runtime that you're casting to the right type 

  Derived* d1_d = static_cast<Derived*>(b1); 
  cout << d1_d << ' ' << d1_d->baseMethod() << ' ' << d1_d->derivedMethod() << endl; 
  // Static (compile-time) downcasts are OK (and faster) if you know you're 
  // casting to the right type 

  Base* b2 = new Base; 
  Derived* d2_d = dynamic_cast<Derived*>(b2); 
  cout << d2_d << endl; 
  // Cast was not valid, so dynamic_cast returned nullptr 
  // If it were a reference, it would've thrown std::bad_cast 

  Derived* d2_s = static_cast<Derived*>(b2); 
  // cout << d2_s << ' ' << d2_s->baseMethod() << ' ' << d2_s->derivedMethod() << endl; 
  // ^ This is undefined, since the cast is wrong (and not checked at runtime) 

  delete b1; // Remember to clean up! 
  delete b2; 
}
  \end{cppcode}
\end{frame}

\begin{frame}[fragile=singleslide]
  \frametitle{Templates}
  \begin{cppcode}
#include <string>
#include <vector>

template<class T>
class Value {
  public:
    Value(const T& data) : _data(data) {}
    const T& getData() { return this->_data; }
  private:
    T _data;
};
// Templates are used A LOT in the standard library

using std::string; 
using std::vector; 

int main() { 
  Value<string> str("My name is Beavis");
  Value<vector<int>> ints({1, 2, 3, 4, 5, 6}); 
  // Two different classes, with two different bits of machine code 
}
  \end{cppcode}
\end{frame}

\begin{frame}[fragile=singleslide]
  \frametitle{Templates}
  \begin{cppcode}
#include <iostream>
#include <array> 
#include <vector> 
#include <string> 

using namespace std; // Slide space constraints 

template<class T> 
void printIt(const T& object) { 
  cout << object << endl; 
} 

template<class Container> 
void needsABeginMethod(Container& container) { 
  auto iterator = container.begin(); // The begin() method returns an iterator 
} 

int main() { 
  printIt(42); // Type parameters in function templates are usually inferred 
  printIt<float>(7.3); // You can specify them explicitly if you need to 

  array<int, 3> ints = {1, 2, 3}; 
  vector<int> vec = {5, 6, 7, 8, 9}; 
  string str = "Strings are containers"; 

  needsABeginMethod(ints); 
  needsABeginMethod(vec); 
  needsABeginMethod(str); 
  // needsABeginMethod(47); ERROR: ints don't have such a method 
}
  \end{cppcode}
\end{frame}

\begin{frame}[fragile=singleslide]
  \frametitle{Templates}
  \begin{cppcode}
#include <array> 
#include <iostream> 
#include <typeinfo> 
#include <string> 

sing namespace std; // slide space limits 

template<class T, int ArrayLength = 10> 
class SomeTypedefs { 
  public: 
    SomeTypedefs() = delete; // Prevent instantiation 

    typedef T Type; 
    typedef T* Pointer; 
    typedef T& Reference; 
    typedef T RawArray[ArrayLength]; 
    typedef array<T, ArrayLength> STLArray; 
};
// The standard library does this a lot ^

int main() {
  cout << typeid(SomeTypedefs<int>::Type).name() << endl;
  cout << typeid(SomeTypedefs<float>::Pointer).name() << endl; 
  cout << typeid(SomeTypedefs<string>::Reference).name() << endl; 
  cout << typeid(SomeTypedefs<int, 14>::RawArray).name() << endl;
  cout << typeid(SomeTypedefs<double, 3>::STLArray).name() << endl; 
}
  \end{cppcode}
\end{frame}

\begin{frame}[fragile=singleslide]
  \frametitle{new}
  \begin{cppcode}
#include <string>

using std::string; 

const int SOME_CONSTANT = 34; 
const string ANOTHER_CONSTANT = "More type-safe than macros!"; 

class ConstStuff { 
  public: 
    int getData() const { 
      // The method doesn't modify the object 
      return this->_data; 
    } 

    const string& getText() const { 
      // Gives you an immutable reference to _text; you must assign it to 
      // a const std::string& if you don't want to copy it 
      return this->_text; 
    } 

    void concatenate(const string& str) { 
      // str itself can't be modified, and it's not copied
      this->_text += str; 
    }

  private: 
    int _data; 
    string _text; 
};
// Attempting to violate const-correctness through runtime tricks is UB
  \end{cppcode}
\end{frame}

\begin{frame}[fragile=singleslide]
  \frametitle{Operator Overloading}
  \begin{cppcode}
#include <iostream> 

using std::cout; 
using std::endl; 
using std::ostream; 

// cout is an ostream 
struct Vector2 { 
  float x, y; 
  Vector2(const float x, const float y) : x(x), y(y) {} 

  Vector2 operator+(const Vector2& other) { 
    return Vector2(x + other.x, y + other.y); 
  } 
}; 

// This is how most types in the standard library are printed out 
ostream& operator<<(ostream& out, const Vector2& v) { 
  return out << '[' << v.x << ", " << v.y << ']';
}

int main() {
  Vector2 a(1, 2); 
  Vector2 b(2, 1); 
  Vector2 c = a + b; 

  cout << c << endl; 
}
  \end{cppcode}
\end{frame}

\begin{frame}[fragile=singleslide]
  \frametitle{String to numbers and back}
  \begin{cppcode}
  \end{cppcode}
\end{frame}

\begin{frame}[fragile=singleslide]
  \frametitle{Data Structures}
  \begin{itemize}
    \item Stuff
  \end{itemize}
\end{frame}

\begin{frame}[fragile=singleslide]
  \frametitle{Random Numbers}
  \begin{cppcode}
  \end{cppcode}
\end{frame}

\begin{frame}[fragile=singleslide]
  \frametitle{String Concatenation}
  \begin{cppcode}
  \end{cppcode}
\end{frame}

\begin{frame}[fragile=singleslide]
  \frametitle{Iterators}
  \begin{cppcode}
  \end{cppcode}
\end{frame}

\begin{frame}[fragile=singleslide]
  \frametitle{Anonymous functions and <algorithm>}
  \begin{cppcode}
  \end{cppcode}
\end{frame}

\begin{frame}[fragile=singleslide]
  \frametitle{Time}
  \begin{cppcode}
  \end{cppcode}
\end{frame}

\begin{frame}[fragile=singleslide]
  \frametitle{Next Week}
  \begin{itemize}
    \item Stuff
  \end{itemize}
\end{frame}

\end{document}